

			% Packages Files
%----------------------------------%

%	% Page Setttings
%\usepackage[top = 1in, bottom = 1in, left = 1in, right = 1in]{geometry}
%
%\parindent=0cm % to prevent the spacing in new paragraph or new line
%
%\usepackage{fancyhdr} % needded for header and footer 

%-----------------------------------------------------------------%

\usepackage{amsmath} % needed for referencing thequation
\usepackage{amsfonts} % for maths symbols
\usepackage{graphicx}
\usepackage[export]{adjustbox}
\usepackage{caption}
\usepackage{refstyle} % to reference a captioned figure


% To make hyperlinks

\usepackage[
	colorlinks=true
	,breaklinks
	]{hyperref} % needed for creating hyperlinks in the document, the option colorlinks=true gets rid of the awful boxes, breaklinks breaks lonkg links (list of figures),
	
\usepackage{xcolor}
\definecolor{c1}{rgb}{0,0,1} % blue
\definecolor{c2}{rgb}{0,0.3,0.9} % light blue
\definecolor{c3}{rgb}{0.3,0,0.9} % red blue
\hypersetup{
    linkcolor={c1}, % internal links
    citecolor={c2}, % citations
    urlcolor={c3} % external links/urls
}

	% For Code Writing
	
\usepackage{listings}

\usepackage{color}

\definecolor{mygreen}{rgb}{0 0.6 0}

\lstset{
% Set automatic Line breaking, in case we have long lines don't
% fit inside the frame or the page.
breaklines = true,
commentstyle = \color{mygreen},
breakatwhitespace = true,% don't show white space character
%numbers = left, % put line number in the code
title = \lstname
}

% ------------- For Writing Equations inisde a table and centering them ------------

\usepackage{array}

% Vertical Alignment inside the cells of the table
\newcolumntype{M}[1]{>{\centering\arraybackslash}m{#1}}

% Horizontal Alignment inside the cells of the table
\newcolumntype{P}[1]{>{\centering\arraybackslash}p{#1}}
